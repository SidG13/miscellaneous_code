\documentclass[11pt]{article}

\usepackage{fullpage}
\usepackage{graphicx}

\begin{document}
\title{Vobii Biology Challenge}
\author{Sid Gopalan}
\maketitle



\section*{Knowledge: Microfilaments}
	Microfilaments are strand-like structures that make up part of the cytoskeleton network within the cell. They are the smallest filament of the cytoskeleton, only 5 - 9 nm in diameter.  
	
	\subsection*{Function}
	Microfilaments are primarily used in cell crawling, adherens junctioning and cytoplasmic division after mitosis. This requires it to be \textbf{dynamic}, assembling and disassembling to perform these functions.
	\subsection*{Structure}
	Microfilaments are made of \textbf{actin monomers}. Two strands (protofilaments) of actin twist in a helix to make a single  filament. Actin has \textbf{plus-minus polarity}, just like microtubules.
	\subsection*{Monomer Addition}
	Free monomers associate with ATP, making actin an \textbf{ATPase} (unlike microtubules, which are GTPases). ATP-associated actin is called  \textbf{T-form} and forms an ATP cap at the plus end, stabilizing the microfilament. Once the monomer is hydrolyzed, it is called \textbf{D-form} (for ADP). In-vitro, \textbf{treadmilling} occurs when the rate of addition at the plus end equals rate of loss at the minus end. At the treadmilling concentration, the size of the filament remains the same, but both ends are constantly being recycled. 
	\subsection*{Motor Protein}
	Because actin is polarized, it has a motor protein, called \textbf{myosin}. Myosin is \textbf{plus-end directed} and hydrolyzes \textbf{ATP}. Among other things, it is responsible for muscle movements and contracts the actin ring during cytoplasmic division. It does this by hooking on to actin and pulling towards the plus end.
\section*{Problem Solving 1: Simple Diffusion}
\textit{Note:} In order to answer any question involving movement across a membrane, it is very helpful to draw a picture to help visualize the problem.
\begin{enumerate}
	\item By looking at the question and our options, we learn that the question involves tonicity. Therefore, the question deals with the  \textbf{simple diffusion} of water. Remember, water naturally diffuses in a direction that \textbf{balances concentration} on both sides of a membrane.
	\item Now, let's understand what happens to cells in solutions of different relative concentrations. 
	\begin{itemize}
		\item \textit{Isotonic} means \underline{equal} concentration of solute on both sides of the membrane. As such, no net movement of water would occur, and \underline{nothing} would happen to the cell.
		\item \textit{Hypotonic} means a \underline{lower} concentration of solute outside the cell versus inside the cell. This means water would enter the cell to balance concentration, causing it to \underline{swell}.
		\item \textit{Hypertonic} means a \underline{higher} concentration of solute outside the cell versus inside the cell. This means water would leave the cell to balance concentration, causing it to \underline{shrink}.
		\begin{center}
\includegraphics[scale=0.19]{Cell.jpg}
\end{center}
	\end{itemize}
	\item Now we can look at our given options and see that only \textbf{(c)} correctly describes what would happen.

\end{enumerate}

\section*{Problem Solving 2: Cell Junctions}
This question is testing our understanding of the different cell junctions and asking us to choose the option that is \textbf{most important to the function of adherens junctions}. Always keep in mind the role, and main transmembrane protein of each junction to solve these types of problems.  

\begin{enumerate}
\item We must first recall some basic information about adherens junctions. Adherens junctions are a subtype of \textbf{anchoring junction}, which form protein links to hold neighbouring cells together. The transmembrane protein that forms this link is called \textbf{cadherin}.
\item Now that we have this information in our minds, the problem becomes easier, and we can go through the options and rule out the ones that are factually incorrect:
	\begin{itemize}
	\item[(a)] Connexin proteins are used in gap junctions, not anchoring junctions. 
	\item[(b)] Adherens junctions link neighbouring cells, not cells to the extracellular matrix. 
	\item[(c)] Claudin is the primary protein that forms tight junctions, not anchoring junctions. 
	\item[\textbf{(d)}] Cadherin is the primary protein that forms adherens junctions, so this must be the correct option.
	\end{itemize}
\end{enumerate} 
\end{document}
